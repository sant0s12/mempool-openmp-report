\chapter{Introduction} \label{ch:introduction}

MemPool~\cite{mempool} is an open-source manycore architecture developed at ETH. It is comprised of
up to 1024~\cite{terapool} 32-bit RISC-V Snitch~\cite{snitch} cores that share a common L1 cache.
Unlike on GPUs, each core is individually programmable, therefore enabling the familiar shared
memory programming model.

Although MemPool's parallelism can be harnessed by manually managing it using the primitives
provided by MemPool's C runtime library, this can become a tedious and error prone task since the
programmer has to be knowledgeable about the underlying architecture. Luckily, higher level
abstraction layers such as OpenMP~\cite{openmp} allow the programmer to express this parallelism in
a clear and concise way using the fork-join paradigm and without needing advanced architectural
knowledge.

OpenMP is comprised of a set of compiler directives for both C/C++ and Fortran that enable this
higher level abstraction. In order to use OpenMP, one has to use a compatible compiler that is able
to roughly do the following:

\begin{enumerate}
	\item The compiler must be able to parse the directives and compile the surrounding code
	      according to their semantics.
	\item The output of the compilation process must match the target platform in order to make use
	      of its parallelism capabilities and primitives.
\end{enumerate}

Both GCC and LLVM provide an OpenMP runtime library interface that bridges the gap between both
points: instead of directly generating a binary from OpenMP annotated code, the compiler generates
calls to the OpenMP runtime library. This library has to be aware of how to parallelize the code
given the platform, which makes it inherently non-portable. Naturally, popular compilers ship with
their own OpenMP runtime library implementation, however, in order to get around the non-portability
issue, they target a host running an operating system which is able to provide the necessary
parallelism primitives. Since MemPool is designed to run bare-metal applications, this means that
a custom OpenMP runtime library has to be developed.

Previously, the authors of MemPool developed a GCC compatible OpenMP runtime library with support
for the most commonly used OpenMP constructs, such as \emph{work sharing}, \emph{critical sections}
and \emph{atomics}.\footnote{This implementation will be discussed in more detail in
\cref{ch:related_work}.} However, as the MemPool team increasingly relies on the LLVM compiler
infrastructure, it is essential to have a compatible OpenMP runtime library for LLVM as well.

The goal of this project is to implement an OpenMP runtime for LLVM that works on MemPool, supports
at least the same set of features as the previous GCC runtime and achieves comparable performance.

The runtime will be implemented in C++ as detailed in \cref{ch:implementation}. In
\cref{ch:results}, the new runtime will be evaluated in terms of performance and correctness against
the GCC runtime. Finally, we will summarize the results and discuss possible shortcomings and future
work in \cref{ch:conclusion}.

% The introduction motivates your work and puts it into a bigger context. It should answer the
% following questions: What is the background of this work? What is the state of the art? Why is
% this project necessary to advance the state of the art? What are the problems that have to be
% solved and why are they difficult? What are your contributions to solve these problems? How do you
% evaluate your solution to show that it is adequate and applicable?
%
% An introduction written along these questions naturally follows the \textit{\gls{spse}}\footnote{%
% The \acrshort{spse} approach was established in a book~\cite{Hoey83}, but is also briefly
% summarized in a more recent article~\cite{MP12}, which is available online. } approach. In the
% \emph{situation}, you set the scene for your work and catch the interest of the readers by showing
% the importance and generality of the scene. In the \emph{problem}, you spot an issue in the scene
% and show why and how it significantly taints the scene. In the \emph{solution}, you outline your
% solution to that issue. Finally, in the \emph{evaluation}, you present the main arguments why the
% claimed solution actually does solve the problem.
%
% In the following chapters, you will elaborate each of the four \gls{spse} elements in detail: In
% \textsl{Background}, you lay the foundations for an in-depth understanding of the situation and
% the problem. In \textsl{Related Work}, you show how others have address this (or similar) problems
% and why their solutions are not sufficient or applicable. In \textsl{Implementation}, you specify
% your solution, which you then evaluate rigorously for strengths and weaknesses in
% \textsl{Results}.
%
% At the end of the introduction, you should explicitly show this structure to the reader by briefly
% explaining how this report is organized. Instead of using the general \gls{spse} terminology and
% the chapter names mentioned above, we urge you to use the domain-specific terminology established
% in the introduction and point to chapters using cross references (e.g., refering to
% \cref{ch:background} instead of ``the Background chapter'').
