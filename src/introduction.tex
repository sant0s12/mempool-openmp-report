\chapter{Introduction}
\label{ch:introduction}

Mempool enables parallelism between its up to 256 cores. If one were to
manually manage this parallelism , it would be a tedious and error prone task.
Luckily, there are high level programming abstractions that allow the
programmer to express this parallelism in a clear and concise way. An
example for this is OpenMP~\cite{openmp}.

OpenMP is comprised of a set of compiler directives for both C/C++ and Fortran
that enable this higher level abstraction. In order to use OpenMP, the compiler
has to support it. Support can be roughly divided into two parts:

\begin{enumerate}
	\item The compiler must be able to parse the directives compile the code surrounding them
	      according to their semantics.
	\item The output of the compilation process must match the target platform
	      in order to make use of its parallelism capabilities and primitives.
\end{enumerate}

For that reason, both GCC and LLVM provide an OpenMP runtime library interface that bridges the gap
between both points: instead of directly generating a binary from OpenMP annotated code, the compiler
generates calls to the OpenMP runtime library. This library has to be aware of how to parallelize the
code given the platform, therefore a new runtime library has to be implemented for each target platform.
Naturally, popular compilers ship with their own OpenMP runtime library implementation, however not
targeting embedded systems.

---

The introduction motivates your work and puts it into a bigger context.
It should answer the following questions:
What is the background of this work?
What is the state of the art?
Why is this project necessary to advance the state of the art?
What are the problems that have to be solved and why are they difficult?
What are your contributions to solve these problems?
How do you evaluate your solution to show that it is adequate and applicable?

An introduction written along these questions naturally follows the \textit{\gls{spse}}\footnote{%
	The \acrshort{spse} approach was established in a book~\cite{Hoey83}, but is also briefly summarized in a more recent article~\cite{MP12}, which is available online.
} approach.
In the \emph{situation}, you set the scene for your work and catch the interest of the readers by showing the importance and generality of the scene.
In the \emph{problem}, you spot an issue in the scene and show why and how it significantly taints the scene.
In the \emph{solution}, you outline your solution to that issue.
Finally, in the \emph{evaluation}, you present the main arguments why the claimed solution actually does solve the problem.

In the following chapters, you will elaborate each of the four \gls{spse} elements in detail:
In \textsl{Background}, you lay the foundations for an in-depth understanding of the situation and the problem.
In \textsl{Related Work}, you show how others have address this (or similar) problems and why their solutions are not sufficient or applicable.
In \textsl{Implementation}, you specify your solution, which you then evaluate rigorously for strengths and weaknesses in \textsl{Results}.

At the end of the introduction, you should explicitly show this structure to the reader by briefly explaining how this report is organized.
Instead of using the general \gls{spse} terminology and the chapter names mentioned above, we urge you to use the domain-specific terminology established in the introduction and point to chapters using cross references (e.g., refering to \cref{ch:background} instead of ``the Background chapter'').
