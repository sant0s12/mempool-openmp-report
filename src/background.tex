\chapter{Background}
\label{ch:background}

{\color{red}
	Explain the background and theory underlying your project.
	Assume that the average reader has the same knowledge you had \emph{before} undergoing this project.
	This chapter should transfer all knowledge necessary to understand the following chapters.

	As this and the following chapters are likely longer than a few pages, consider structuring them into sections (but avoid fragmentation by overly fine-grained sectioning).
	Use the \verb|\..section{}| command family as illustrated below:

	\section{First section}
	\label{sec:background_overview}

	\subsection{First subsection}

	\subsection{Second subsection}

	\subsubsection{First subsubsection}

	Follow a top-down approach when structuring the chapter, and guide the reader by giving a short overview at the beginning of each section.
	Again, you can use labels and references (e.g., referring to \cref{sec:background_overview}).
}
