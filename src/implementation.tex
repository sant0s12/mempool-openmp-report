\chapter{Implementation}
\label{ch:implementation}

\section{General Architecture}
\label{sec:general-architecture}

The OpenMP runtime library presented in this report is implemented in C++, as opposed to the
previous version, which was implemented in C. The main reason for this is that C++ provides a higher
level of abstraction compared to C, allowing us to use features such as \emph{classes},
\emph{templates}, and some parts of the standard library that don't require OS support, such as
\emph{lock guards}. If used properly, C++ should have no meaningful performance overhead compared to
C, but we will look into this in more detail in \autoref{ch:results}. Another reason is that, until
now, C++ was only used in MemPool in the context of Halide~\cite{halide}, so we use this as an
opportunity to explore how to enable C++ support in MemPool.

The runtime is roughly structured into the following three main components:

\begin{itemize}
	\item KMP entrypoints.
	\item C++ classes for each relevant concept (e.g., threads, teams, barriers, etc.).
	\item Supporting code.
\end{itemize}

Each of them will be described in more detail in the following sections.

\section{KMP Entrypoints}

The KMP entrypoints are the functions that are called at runtime during the execution of an OpenMP
program. During the compilation process, the compiler converts OpenMP directives into calls to these
functions as defined by the internal \gls{api} between the LLVM compiler and the runtime library. An
LLVM OpenMP runtime reference document is available at~\cite{kmpref}, however it is quite out of
date and not very helpful in documenting the expected behavior of the runtime library. For this
reason, using the source code of the default runtime
library\footnote{\url{https://github.com/llvm/llvm-project/tree/main/openmp}}, as well as smaller
third-party implementations\footnote{\url{https://github.com/parallel-runtimes/lomp}} can be quite
helpful.

In order to maintain some separation of concerns, the functionality associated with each entrypoint
is not directly implemented in the function, but rather in a one of the different classes described
in the next chapter, which contains the relevant methods associated with the entrypoint. Those are
then called by the entrypoint function.

Each implemented entrypoint is documented in \cref{sec:kmp-entrypoints}.

\section{C++ Classes}

The runtime library is structured around a set of C++ classes, each representing a different high
level concept, which allows us to compartmentalize the implementation.

The main classes are described in the following sections.

\subsection{Task}

The \texttt{Task} class is essentially a wrapper around a KMP \texttt{microtask}, which represents a
section of code to be run by a thread (e.g., the body of a \texttt{parallel} section) and is passed
in as an argument to, e.g., \texttt{__kmpc_fork_call} (\cref{subsubsec:kmpc-fork-call}).
Additionally, this class also stores the arguments that must be passed to the \texttt{microtask}.
These are pointers used by the compiler to access data outside of the \texttt{microtask}'s scope.

Documentation for this class can be found in \cref{subsec:task}.

\subsection{Barrier}

This class is used to implement a thread barrier. It provides two implementations: one of them is
more performant but only works when there is only a single team running, whereas the other one works
with multiple teams.

Documentation for this class can be found in \cref{subsec:barrier}.

\subsection{Thread}

This class represents a single OpenMP thread running on a core and contains attributes such as the
thread ID and the team it belongs to, as well as methods such as the main event loop equivalent to
\cref{lst:event-loop} and the logic for executing a fork.

Documentation for this class can be found in \cref{subsec:thread}.

\subsection{Team}

This class represents a team of threads. It contains attributes such as the team ID and the number
of threads participating in it, as well as the task to be executed by all threads in the team, and
the barrier associated with it. Additionally, it contains methods for assigning work to the threads
and waking them up, as well as for scheduling static and dynamic work shares.

Documentation for this class can be found in \cref{subsec:team}.

\section{Supporting Code}

The supporting code consists of various helper functions and global variables in the
\texttt{runtime} namespace (\cref{subsec:runtime-namespace}) used throughout the runtime, as well as
as everything surrounding C++ support, and the main function wrapper. In the following sections, we
will describe the latter two.

\subsection{C++ Support}

\subsubsection{Heap Allocation}

In order to support heap allocation in C++, we implement the global \texttt{operator new} and
\texttt{operator delete} functions since they are used by the default
allocator\footnote{\url{https://en.cppreference.com/w/cpp/memory/allocator}}. They are essentially
wrappers around the \texttt{simple\_malloc} and \texttt{simple\_free} functions provided by the
MemPool runtime library. Additionally, they use a globally defined mutex \texttt{allocLock} to make
sure that only one thread uses the allocator at the same time, since neither \texttt{simple\_malloc}
nor \texttt{simple\_free} are thread-safe.

\begin{lstlisting}[language=C, caption={Heap Allocation Functions},
	label={lst:heap-allocation}, escapechar=@, literate={~} {$\sim$}{1}]
kmp::Mutex allocLock __attribute__((section(".l1")));

void *operator new(size_t size) {
  std::lock_guard<kmp::Mutex> lock(allocLock);
  void *ptr = simple_malloc(size);
  return ptr;
}

void operator delete(void *ptr) noexcept {
  std::lock_guard<kmp::Mutex> lock(allocLock);
  return simple_free(ptr);
}

void *operator new[](size_t size) {
  return operator new(size);
}

void operator delete[](void *ptr) noexcept {
  return operator delete(ptr);
}
\end{lstlisting}

\subsubsection{Compiler/Linker Flags}

The C++ code is compiled with the following compiler flags:
\begin{itemize}
	\item \texttt{-std=c++17} to enable C++17 features.
	\item \texttt{-fno-exceptions} to disable exception handling.
	\item \texttt{-fno-threadsafe-statics} to disable thread-safe initialization of static
	      variables, since they are not used.
\end{itemize}

And linked with the following linker flags:
\begin{itemize}
	\item \texttt{-nostdlib} to not link against the standard C library because it does not support
	      the architecture.
\end{itemize}

\subsubsection{Global Variables}

Global C++ objects need to be initialized at runtime. Usually, this would be done by the startup
files that the program is liked to by default, however, they are disabled when using the
\texttt{-nostdlib} flag. This is why we need to implement this initialization ourselves. The
compiler generates an array of initialization functions~\cite{crt-startup} in the
\texttt{.init\_array} section of the \gls{elf} file, each of which should be called before the main
function. This is done by the \texttt{initGlobals} function (\cref{lst:init-globals}), in
conjunction with the additions to the linker script shown in \cref{lst:init-array-linker}.

\begin{lstlisting}[language=C, caption={initGlobals},
	label={lst:init-globals}, escapechar=@, literate={~} {$\sim$}{1}]
typedef void (*init_func)(void);
extern init_func __init_array_start[];
extern init_func __init_array_end[];

static inline void initGlobals() {
  int32_t len = __init_array_end - __init_array_start;
  for (int32_t i = 0; i < len; i++) {

    __init_array_start[i]();
  }
}
\end{lstlisting}

\begin{lstlisting}[language=C, caption={Init Array Linker Script},
	label={lst:init-array-linker}, escapechar=@, literate={~} {$\sim$}{1}]
/* Init array on L2 */
.init_array : {
    HIDDEN (__init_array_start = .);
    KEEP (*(SORT_BY_INIT_PRIORITY(.init_array.*)))
    KEEP (*(.init_array))
    HIDDEN (__init_array_end = .);
} > l2
\end{lstlisting}

\subsection{Main Function Wrapper}

In order to not have to write the code from \cref{lst:event-loop} manually in every OpenMP program,
we implement a wrapper around the \texttt{main} function by compiling them with the \texttt{-wrap
	main} linker flag, which renames the \texttt{main} function to \texttt{\_\_real\_main} and
\texttt{\_\_wrap\_main} to \texttt{main}. This means that the branch depending on the core ID is
done transparently to the user. Additionally, we can also perform initialization tasks before
calling the real \texttt{main} function, such as initializing the MemPool heap allocators and
initializing global variables.

\begin{lstlisting}[language=C, caption={Main Function Wrapper},
	label={lst:main-wrapper}, escapechar=@, literate={~} {$\sim$}{1}]
extern "C" int __real_main();

bool initLock = true;

extern "C" int __wrap_main() {
  const mempool_id_t core_id = mempool_get_core_id();
  if (core_id == 0) {
    // Init heap allocators
    mempool_init(0);

    // Call C++ global constructors
    initGlobals();

    initLock = false;

    // Run the program
    __real_main();

    printf("Program done\n");
  } else {
    while (initLock) {
      // Wait for initialization to finish
    }

    kmp::runtime::runThread(static_cast<kmp_int32>(core_id));
  }

  return 0;
}
\end{lstlisting}
