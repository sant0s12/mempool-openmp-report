\chapter*{Abstract}

In order to effectively harness MemPool's parallelism, an initial GCC-compatible OpenMP runtime was
developed for this architecture, supporting the most commonly used set of features. However, due to
increasing reliance on the LLVM compiler infrastructure by the MemPool team, it is becoming crucial
to have an OpenMP runtime compatible for it. This project aims to implement an OpenMP runtime that
works with LLVM, supporting at least the same set of features as the previous runtime, while
maintaining comparable performance and offering the possibility of future additions and
improvements.\todo{Add examples for improvements, perhaps cite the dynamic memory remapping thing}

% The abstract summarizes what this report is about.
% It focusses on the big picture and does not go into details.
% You should write concisely about the following points:
%
% \begin{itemize}
% 	\item Describe the \textbf{background} of your project: what is the motivation for your project and why is it important?
% 	\item Describe the \textbf{objectives} of your project.
% 	\item Describe the \textbf{problems} that must be addressed to achieve the objectives---why are these problems difficult?
% 	\item Describe your \textbf{approach} and \textbf{methods}.
% 	\item Summarize the most important \textbf{results}.
% 	\item State the main \textbf{conclusion} and its significance.
% \end{itemize}
%
% The abstract typically takes half a page and should not be longer than one full page.
% Try to write a draft of the abstract early on to have a good idea of your project, but revise the abstract as the project progresses.
% Write the final version of the abstract once the report is otherwise complete.
%
% The remainder of this document contains an example on structure and content of the report.
% This template is meant to guide you and not to force you into a certain structure---just make sure you and your advisors agree on content and structure of the report \emph{before} you start writing it.
% \Cref{app:topic-specific_guidelines} gives more specific guidelines for some major project areas (e.g., hardware designs).
% If you are new to \LaTeX{} or want to learn some best practices, you should also check the short \LaTeX{} guide in \cref{app:LaTeX_guide}.
